\documentclass{article}
\usepackage[utf8]{inputenc}
\usepackage[english]{babel}
\usepackage{indentfirst}
\usepackage{graphicx}
\usepackage{xcolor}
\usepackage{hyperref}
\usepackage{tocloft}
\usepackage{geometry}

\newcommand{\cmd}[1]{\texttt{#1}}
\newcommand{\ib}[1]{\textit{\textbf{#1}}}

\hypersetup{
	colorlinks = true,
	linkbordercolor = 1 1 1,
	urlcolor = blue,
	linkcolor = black,
	citecolor = black,
	urlbordercolor = 1 1 1
}

\pagestyle{headings}

\geometry{a4paper, total={160mm,240mm}, left=25mm, top=35mm}

\title{\textbf{STM - Jeu sérieux\\Partie 1 - Rapport}}
\author{Cédric Pendville, Dany A. Darghouth}
\date{}

\renewcommand{\cftsecafterpnum}{\hspace*{7.5em}}
\renewcommand{\cftsubsecafterpnum}{\hspace*{7.5em}}
\renewcommand{\cftsecindent}{7.5em}
\renewcommand{\cftsubsecindent}{7.5em}
\addto\captionsenglish{\renewcommand*\contentsname{\hspace{4.5em}Sommaire}}

\begin{document}

\maketitle

% \tableofcontents

% \newpage
\section{Description du projet}
Dans le cadre du projet de serious game à developper avec Unity, Notre proposition consiste en un Jeu basé sur les LEGO, qui aura pour but de faire découvrir aux enfants les bases de la logique et de la programmation.\\ 

Le jeu se complexifiera au fur et à mesure que le joueur avance dans les niveaux. En proposant des problèmes à résoudre en uitlisant des circuit et portes logiques de plus en plus complexes. Et dans les cas où le joueur se trouve en difficulté des aides pourraient lui être proposées.\\

Ce jeu sera développé en C\# avec le moteur de jeu Unity qui proose un template pour les jeux Lego et qui sera donc un bon point de départ pour notre projet.\\

\end{document}
\[\]